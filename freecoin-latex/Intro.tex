\chapter{Introdução}
\label{chap:int}

%% Para fazer um mini indice do capitulo abrir o ficheiro ``formatacaoUBI.tex" e procurar ''%% O código seguinte permite gerar um mini indice de capitulo (não referido no despacho reitoral)''
%\minitoc


\section{Descrição do Problema}
Com cada vez mais o uso das novas tecnologias, cada vez mais existe a necessidade da existência de moedas criptográficas. Para a cria uma moeda eletrónica, todas as comunicações devem ser cifradas com chaves públicas e privadas de forma a manter a autenticidade do utilizador, para que a troca de mensagens seja segura. Por outro lado, para a criação da moeda existem tarefas que são enviadas por parte do utilizador onde o utilizador que resolver esse problema irá ser compensado. Os utilizadores, podem ainda efetuar transações entre eles sempre com o conhecimento do servidor que controla todo o dinheiro eletrónico que existe no sistema e em cada utilizador.


A aplicação, fornece assim mecanismos para a criação de um utilizador com a criação das suas chaves publica e privada e todas as mensagens são então cifradas com as chaves (tanto do servidor como do utilizador), o que faz com que haja autenticidade de ambas as partes. A aplicação é capaz de enviar com um tempo definido as tarefas a serem tratadas pelos utilizadores e receber as suas respostas de forma a outros utilizadores poderem parar de executar essa tarefa.


\section{Constituição do Grupo}
O grupo é constituído pelos seguintes discentes de Mestrado de Engenharia Informática da Universidade da Beira Interior, sendo estes:

\begin{itemize}
    \item Guilherme Calado Boino-M8468
    
    \item João Alexandre de Aguiar Amaral dos Santos-M8810
    
    \item Nelson Ricardo Matos Fonseca-M8351
    
    \item Vasco Ferrinho Lopes-M8486
    
\end{itemize}




\section{Organização do Documento}

De modo a refletir o trabalho que foi feito, este documento encontra-se estruturado da seguinte forma:
\begin{enumerate}
    %\item O primeiro capítulo -- \textbf{Resumo} -- faz uma breve descrição dos objetivos do trabalho, bem como a abordagem tomada para alcançar os mesmos.
    \item O primeiro capítulo -- \textbf{Introdução} -- descreve o problema e a constituição do grupo, bem como a organização do documento.
    
    \item O segundo capítulo -- \textbf{Engenharia de Software e da Segurança} -- faz uma análise crítica aos requisitos e aos caos de uso, bem como uma abordagem ao modelos do sistema e dos ataques que podem ou não ser efetuados.
    
    \item O terceiro capítulo -- \textbf{Implementação} -- aqui são descritos detalhes técnicos acerca da implementação de código, algoritmos utilizados e manuais de instalação e configuração.
   
    \item O quarto capítulo -- \textbf{Testes ao Sistema} -- enumera os testes de segurança e dos resultados obtidos para os testes realizados.
    
    \item O quinto capítulo -- \textbf{Reflexão Crítica e Problemas Encontrados} -- enumera os objetivos alcançados, bem como os problemas encontrados e a divisão de tarefas efetuada.
    
    \item O sexto capítulo -- \textbf{Conclusões e Trabalho Futuro} -- são tiradas as ilações finais acerca de todo o desenvolvimento do projeto.
\end{enumerate}