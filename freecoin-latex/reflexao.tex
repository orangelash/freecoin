\chapter{Reflexão Crítica e Problemas Encontrados}
\label{chap:reflexao}

\section{Objetivos Propostos vs Alcançados}
\label{sec:objetivos}
O principal objetivo deste trabalho prático foi a elaboração de um programa em CLI onde fosse permitida a ligação através de um par de chaves pública e privada para que dessa forma seja efetuadas conceções para uma prova de trabalho (desta forma podem obter fr€coin) e efetuar transações entre os vários utilizadores do sistema.


Os vários objetivos que foram estabelecidos para o sistema fosse implementado de forma correta e com as funcionalidades mais básicas foram:
\begin{enumerate}
    
    \item O registo de um novo utilizador é feito de forma segura.
    
    \item O par de chaves são gerados do lado do utilizador.
    
    \item Todas as comunicações entre utilizadores e entidade central são feitas de forma segura (e.g., cifradas e protegidas por mecanismos de integridade).
    
    \item O sistema permite as transações conforme descritas na breve descrição.
    
    \item O sistema implementa a funcionalidade de geração de novas moedas.

\end{enumerate}
    
    
    !!!!!!!!!!!!!!!!ESCREVER APOS ESTAR REALIZADO!!!!!!!!!!!!!!!!!!!!!!!!!!
    
    Depois de todos estes objetivos descritos anteriormente serem cumpridos, avançamos então para os objetivos extra que foram propostos inicialmente de forma a melhorar todo o sistema e acrescentar conhecimento. Os objetivos propostos foram:

\begin{enumerate}
    \item O sistema suporta transações completamente anónimas entre utilizadores (i.e., geralmente de um par de chaves por transação).
    
    \item O sistema suporta autenticação mútua (cliente e servidor).
    
    \item A entidade central controla a emissão da moeda, ajustando o grau de dificuldade do problema à velocidade como a rede o envolve.
    
    \item A entidade central é também uma autoridade certificadora, e o sistema passa a estar suportado por uma infraestrutura de chave pública.
    
    \item O sistema é implementado com criptografia sobre curvas elípticas.
    
    \item Outras funcionalidades relevantes no contexto da segurança do sistema e que o favoreçam na nota.!!!!!!!!!!!!!!!!!!!!!!!!!!!!!
\end{enumerate}    
    
    
    !!!!!!!!!!!!!!!!!!!ESCREVER SOBRE OS EXTRAS!!!!!!!!!!!!!!!!!!!!!!!
    
\section{Divisão de Trabalhos pelos Elementos do Grupo}
\label{sec:divisao}
O trabalho foi realizado com todos os elementos do grupo presentes, sendo implementada cada funcionalidade com o conhecimento de todos os clientes do grupo. No entanto, foram distribuídas tarefas de forma a aumentar a produtividade do mesmo.


Numa fase inicial, foi discutido entre todos os membros do grupo a implementação quanto à estrutura das comunicações e a forma como se poderia tornar estas mais seguras.

!!!!!!!!!!!!!!!!!!!!CONTINUAR CONFORME FOR FEITO O TRABALHO!!!!!!!!!!!!!!!!



\section{Problemas Encontrados e Reflexão Crítica}
\label{sec:problemas}

\section{Conclusão}
\label{sec:conclusaoReflexao}