\documentclass[10pt,twoside]{estiloUBI}




\include{formatacaoUBI}

\usepackage{fontspec}


%%Comentar a linha seguinte se escrever a tese em inglês
\portugues


%%Para índice remissivo
\makeindex


%%Escolher tipo de letra a usar:
%\usepackage{lmodern}												%Latin modern
%\usepackage{palatino}												%Palatino
%\usepackage{times}												    %Times


%%O comando seguinte insere o nome da tese no cabeçalho das páginas (comentar se não for pretendido)
\cabecalho{Moeda Criptográfica Controlada Centralmente}



\begin{document}

%%O comando seguinte insere o espaçamento de 1.5 linhas
\onehalfspacing

%%Página de rosto
\pagenumbering{roman}


%\dominitoc


%%Numeração das páginas
\pagestyle{fancy}


%%O comando a seguir gera uma página após a de rosto com cabeçalho e rodapé
% \cleardoublepage

%%O comando a seguir permite que as costas da página de rosto não inclua cabeçalho mas rodapé (escolher entre este e outro)
%\newpage\mbox{}\thispagestyle{plain}\fancyhead{}


%%Dedicatória

%\newpage 	
%\mbox{}
%\vfil
%\begin{center}
%Dedicated to...
%\end{center}
%\vfil
%\eject
%\cleardoublepage


%%Agradecimentos 



%%Prefácio 



%%Resumo+palavras-chave


%%Resumo alargado 


%%abstract+keywords


%%Índice

%%Lista de figuras
{\begin{titlepage}
\begin{center}

\begin{flushleft}
 \includegraphics[height=2.22cm]{logo}\\
\rostoubi UNIVERSIDADE DA BEIRA INTERIOR\\
\rostofac Engenharia\\
\end{flushleft}

\vspace{5.6cm}

\rostotitulo \textbf{ Moeda Criptográfica Controlada Centralmente} \\
\rostosubtit \textbf{Fr€coin}\\

\vspace{1.8cm}

\rostonomes \textbf{Guilherme Calado Boino, M8468\\
João Alexandre de Aguiar Amaral dos Santos, M8810 \\
Nelson Ricardo Matos Fonseca, M8351\\
Vasco Ferrinho Lopes, M8486}\\

\vspace{1.4cm}

\rostooutros Trabalho Prático Sistemas Software Seguros\\
\rostonomes \textbf{Engenharia Informática}\\
\rostooutros (2º ciclo de estudos)\\

\vspace{3.3cm}

\rostooutros Orientador: Prof. Doutor Pedro Ricardo Morais Inácio\\
%Co-orientador: Prof. Doutor Nome\\

\vspace{1.4cm}

\rostooutros \textbf{Covilhã, Maio de 2018}

\end{center}
\end{titlepage}

 \let\cleardoublepage\clearpage}


\newpage \let\cleardoublepage\clearpage
\section*{\titulos{Resumo}}
Com o aparecimento e crescimento que tecnologias como a Bitcoin tiveram nos últimos anos, e devido às suas características criptgráficas, torna-se interessante abordar a sua constituição. Assim, introduzimos a Fr€€Coin, uma moeda criptográfica controlada centralmente.
A Fr€€Coin baseia-se num sistema central de confiança que controla a quantidade existente da moeda e todas as transações, além de controlar a atribuição de moedas a partir de desafios feitos aos utilizadores, de dificuldade variável.

!!!!!!!!!!!!!!!!!!!!!!!!!!!!!!!!


{\tableofcontents \let\cleardoublepage\clearpage
\listoffigures \let\cleardoublepage\clearpage
\listoftables \let\cleardoublepage\clearpage}

% \cleardoublepage


%%Abreviaturas
\newpage \let\cleardoublepage\clearpage
\section*{\titulos{Lista de Acrónimos}}
\vspace{0.5cm}
  \begin{tabularx}{\linewidth}{c p{0.5cm} Y}
 	UBI & & Universidade da Beira Interior\cr
 	MPSOCD & & Multi-objective Particle Swarm Optimization Crowding Distance
 	\end{tabularx}
%  \cleardoublepage
  

%% Os capitulos são inseridos a partir daqui 
 
\mainmatter

{\chapter{Introdução}
\label{chap:int}

%% Para fazer um mini indice do capitulo abrir o ficheiro ``formatacaoUBI.tex" e procurar ''%% O código seguinte permite gerar um mini indice de capitulo (não referido no despacho reitoral)''
%\minitoc


\section{Descrição do Problema}
Com cada vez mais o uso das novas tecnologias, cada vez mais existe a necessidade da existência de moedas criptográficas. Para a cria uma moeda eletrónica, todas as comunicações devem ser cifradas com chaves públicas e privadas de forma a manter a autenticidade do utilizador, para que a troca de mensagens seja segura. Por outro lado, para a criação da moeda existem tarefas que são enviadas por parte do utilizador onde o utilizador que resolver esse problema irá ser compensado. Os utilizadores, podem ainda efetuar transações entre eles sempre com o conhecimento do servidor que controla todo o dinheiro eletrónico que existe no sistema e em cada utilizador.


A aplicação, fornece assim mecanismos para a criação de um utilizador com a criação das suas chaves publica e privada e todas as mensagens são então cifradas com as chaves (tanto do servidor como do utilizador), o que faz com que haja autenticidade de ambas as partes. A aplicação é capaz de enviar com um tempo definido as tarefas a serem tratadas pelos utilizadores e receber as suas respostas de forma a outros utilizadores poderem parar de executar essa tarefa.


\section{Constituição do Grupo}
O grupo é constituído pelos seguintes discentes de Mestrado de Engenharia Informática da Universidade da Beira Interior, sendo estes:

\begin{itemize}
    \item Guilherme Calado Boino-M8468
    
    \item João Alexandre de Aguiar Amaral dos Santos-M8810
    
    \item Nelson Ricardo Matos Fonseca-M8351
    
    \item Vasco Ferrinho Lopes-M8486
    
\end{itemize}




\section{Organização do Documento}

De modo a refletir o trabalho que foi feito, este documento encontra-se estruturado da seguinte forma:
\begin{enumerate}
    %\item O primeiro capítulo -- \textbf{Resumo} -- faz uma breve descrição dos objetivos do trabalho, bem como a abordagem tomada para alcançar os mesmos.
    \item O primeiro capítulo -- \textbf{Introdução} -- descreve o problema e a constituição do grupo, bem como a organização do documento.
    
    \item O segundo capítulo -- \textbf{Engenharia de Software e da Segurança} -- faz uma análise crítica aos requisitos e aos caos de uso, bem como uma abordagem ao modelos do sistema e dos ataques que podem ou não ser efetuados.
    
    \item O terceiro capítulo -- \textbf{Implementação} -- aqui são descritos detalhes técnicos acerca da implementação de código, algoritmos utilizados e manuais de instalação e configuração.
   
    \item O quarto capítulo -- \textbf{Testes ao Sistema} -- enumera os testes de segurança e dos resultados obtidos para os testes realizados.
    
    \item O quinto capítulo -- \textbf{Reflexão Crítica e Problemas Encontrados} -- enumera os objetivos alcançados, bem como os problemas encontrados e a divisão de tarefas efetuada.
    
    \item O sexto capítulo -- \textbf{Conclusões e Trabalho Futuro} -- são tiradas as ilações finais acerca de todo o desenvolvimento do projeto.
\end{enumerate} \let\cleardoublepage\clearpage
% \cleardoublepage
\chapter{Engenharia de Software e da Segurança}
\label{chap:engsoft}

\section{Análise dos Requisitos e Casos de Uso}
\label{sec:analreq}
De forma ao sistema corresponder a todas as expectativas, foi elaborada uma lista de requisitos composta por:

\begin{enumerate}
    \item Deve ser efetuadas comunicações(seguras) cliente-servidor e servidor-cliente.
    
    \item O utilizador deve realizar um registo inicial onde é gerado um par de chaves (pública e privada).
    
    \item O sistema deve permitir transações entre os seus clientes.
    
    \item O sistema deve lançar desafios de forma a criar novas moedas(fr€coin).
    
    \item O sistema deve suportar transações completamente anónimas entre utilizadores.
    
    \item A entidade central é também autoridade certificadora(servidor).
    
    \item O sistema deve ser implementado sobre curvas elípticas.
    
    \item O sistema deve suportar efetuar operações em simultâneo de forma a efetuar trabalho.
    
    \item O sistema deve lançar desafios sobre um tempo pré definido.
    
    \item O sistema deve ser capaz de controlar o número de fr€coin existente.
    
    \item O servidor deve ser uma entidade de confiança.
    
    \item O servidor deve conter informação acerca do nome do utilizador, a sua chave publica !!!!MAIS!!!!.
    
    \item O servidor deve conter todos os documentos das transações assinados pelas entidades que as efetuam.
    
    \item O cliente deve conter o seu par de chaves (pública e privada) e a chave pública do servidor.
    
\end{enumerate}

Quanto aos casos de uso, estes podem ser da forma:
!!!!!!!!!!!!!!!!!!!VERIFICAR DIAGRAMAS E MUDAR, ESTA NO PC BOINO OS DIAGRAMAS!!!!!!!!!!!!!!!!!!!


\begin{figure}[H]
\centering
\includegraphics[width=.7\textwidth]{imagens/login.png}
\caption{Diagrama de login.}
\label{fig7}
\end{figure}


Sempre que se efetua o login, o utilizador envia um ficheiro cifrado com a sua chave privada, quando este pacote chega ao utilizador este verifica a assinatura e então é efetuado o login para esse cliente.

\begin{figure}[H]
\centering
\includegraphics[width=.7\textwidth]{imagens/Registar.png}
\caption{Diagrama de Registar.}
\label{fig7}
\end{figure}


Neste diagrama, podemos verificar que o cliente irá efetuar uma comunicação com o servidor após este enviar uma mensagem encriptada. Ao chegar ao servidor é verificado e guardada o nome do utilizador e a sua chave pública correspondente, desta forma irá enviar uma resposta ao cliente e a partir desse momento podem assim efetuar uma ligação segura.

\begin{figure}[H]
\centering
\includegraphics[width=.7\textwidth]{imagens/Transação.png}
\caption{Diagrama de transação.}
\label{fig7}
\end{figure}


Sempre que se pretende efetuar uma transação, o cliente envia uma mensagem ao servidor, assinada com a sua chave privada. Ao chegar ao servidor, este verifica a sua assinatura e se o cliente contem verba suficiente para esta transação. Se o cliente tiver esse montante descrito, é então enviada uma mensagem ao cliente 2(recetor) e é assinado por este. Caso não contenha a verba suficiente o cliente 2 apenas irá verificar a mensagem no entanto não a irá assinar. 
Todas as transações devem passar pelo servidor e com isso serem todas armazenadas no mesmo para mais tarde poderem ser verificadas.


\section{Outros Diagramas}
\label{sec:diagramas}


!!!!!!!!!!FAZER MODELOS DE ATAQUE??????ALGUNS BASICOS??????????!!!!!!!!!!

\section{Modelo do Sistema, Modelos de Ataque e propriedades de Segurança}
\label{sec:modelos}

\section{Modelação de Ataques}
\label{sec:ataques}

\section{Conclusão}
\label{sec:conclusao}

 \let\cleardoublepage\clearpage
\chapter{Implementação}
\label{chap:implmentacao}

\section{Ferramentas e Tecnologias utilizadas}
\label{sec:ferramentas}
O projeto foi desenvolvido de forma a ser usado num computador, para tal foi utilizada a linguagem \textit{Java}, compilada e testada no IDE \textit{NetBeans}. De forma a se encontrar disponível para todos os discentes podendo trabalhar em simultâneo e efetuando controlo de versões.


Desta forma, resulta uma aplicação que corre em linha de comandos (CLI) e um relatório concebido em \emph{Share} \LaTeX



\section{Escolhas de Implementação}
\label{sec:escolhas}

Numa fase inicial, foi necessário tomar algumas decisões fulcrais. Algumas destas decisões basearam-se na escolha da linguagem a ser usada, desta forma foi escolhida a linguagem \textit{Java} derivado a ser uma linguagem orienteada a objetos e também com a criação de uma interface CLI.

\section{Detalhes de Implementação}
\label{sec:detalhes}

\section{Manual de Instalação}
\label{sec:instalacao}

\section{Manual de Utilização}
\label{sec:utilizacao}

\section{Conclusão}
\label{sec:conc} \let\cleardoublepage\clearpage
\chapter{Testes ao Sistema}
\label{chap:testes}

\section{Testes de Segurança}
\label{sec:seguranca}

\section{Resultados}
\label{sec:resultados}

\section{Conclusão}
\label{sec:conclusTestes} \let\cleardoublepage\clearpage
\chapter{Reflexão Crítica e Problemas Encontrados}
\label{chap:reflexao}

\section{Objetivos Propostos vs Alcançados}
\label{sec:objetivos}
O principal objetivo deste trabalho prático foi a elaboração de um programa em CLI onde fosse permitida a ligação através de um par de chaves pública e privada para que dessa forma seja efetuadas conceções para uma prova de trabalho (desta forma podem obter fr€coin) e efetuar transações entre os vários utilizadores do sistema.


Os vários objetivos que foram estabelecidos para o sistema fosse implementado de forma correta e com as funcionalidades mais básicas foram:
\begin{enumerate}
    
    \item O registo de um novo utilizador é feito de forma segura.
    
    \item O par de chaves são gerados do lado do utilizador.
    
    \item Todas as comunicações entre utilizadores e entidade central são feitas de forma segura (e.g., cifradas e protegidas por mecanismos de integridade).
    
    \item O sistema permite as transações conforme descritas na breve descrição.
    
    \item O sistema implementa a funcionalidade de geração de novas moedas.

\end{enumerate}
    
    
    !!!!!!!!!!!!!!!!ESCREVER APOS ESTAR REALIZADO!!!!!!!!!!!!!!!!!!!!!!!!!!
    
    Depois de todos estes objetivos descritos anteriormente serem cumpridos, avançamos então para os objetivos extra que foram propostos inicialmente de forma a melhorar todo o sistema e acrescentar conhecimento. Os objetivos propostos foram:

\begin{enumerate}
    \item O sistema suporta transações completamente anónimas entre utilizadores (i.e., geralmente de um par de chaves por transação).
    
    \item O sistema suporta autenticação mútua (cliente e servidor).
    
    \item A entidade central controla a emissão da moeda, ajustando o grau de dificuldade do problema à velocidade como a rede o envolve.
    
    \item A entidade central é também uma autoridade certificadora, e o sistema passa a estar suportado por uma infraestrutura de chave pública.
    
    \item O sistema é implementado com criptografia sobre curvas elípticas.
    
    \item Outras funcionalidades relevantes no contexto da segurança do sistema e que o favoreçam na nota.!!!!!!!!!!!!!!!!!!!!!!!!!!!!!
\end{enumerate}    
    
    
    !!!!!!!!!!!!!!!!!!!ESCREVER SOBRE OS EXTRAS!!!!!!!!!!!!!!!!!!!!!!!
    
\section{Divisão de Trabalhos pelos Elementos do Grupo}
\label{sec:divisao}
O trabalho foi realizado com todos os elementos do grupo presentes, sendo implementada cada funcionalidade com o conhecimento de todos os clientes do grupo. No entanto, foram distribuídas tarefas de forma a aumentar a produtividade do mesmo.


Numa fase inicial, foi discutido entre todos os membros do grupo a implementação quanto à estrutura das comunicações e a forma como se poderia tornar estas mais seguras.

!!!!!!!!!!!!!!!!!!!!CONTINUAR CONFORME FOR FEITO O TRABALHO!!!!!!!!!!!!!!!!



\section{Problemas Encontrados e Reflexão Crítica}
\label{sec:problemas}

\section{Conclusão}
\label{sec:conclusaoReflexao} \let\cleardoublepage\clearpage
\chapter{Conclusões e Trabalho Futuro}
\label{chap:conclusao}

\section{Conclusões Principais}
\label{sec:principais}

\section{Trabalho Futuro}
\label{sec:futuro}
 \let\cleardoublepage\clearpage

\phantomsection
\addcontentsline{toc}{chapter}{Bibliografia}
\bibliographystyle{estilo-biblio}				%Estilo bibliografia com nomes
\bibliography{bibliografia}	
}





%% Fim da inserção dos capitulos


%% Inicio Bibliografia
%\cleardoublepage

%%%%%%%%%%%%%%%%
% Escolher entre as duas opcções
%
% A primeira é a aconselhada pelo despacho reitoral
% A segunda é a utilizada pelo IEEE
%
%Primeira opcção
				%Entrada biblbiografia aconselhada com nomes
%
% Segunda opcção
%\bibliographystyle{IEEEtran}					%Estilo bibliografia IEEE
%\bibliography{IEEEabrv,bibliografia}				%Entrada bibliografia aconselhada para IEEE
%% Fim Bibliografia


%%Anexos
% \appendix
 
% \include{Anexos}
% \cleardoublepage


% %%Glossário
% \newpage
% \section*{\titulos{Glossário}}
% \vspace{0.5cm}
% 	\noindent\begin{tabularx}{\linewidth}{l p{0.5cm} Y}
% 	\LaTeX & & Conjunto de macros para o processador de textos \TeX, utilizado amplamente para a produção de textos matemáticos e científicos devido à sua alta qualidade tipográfica.\cr
% 	\end{tabularx}
% \cleardoublepage



%%Inserir índice remissivo
% \printindex

\end{document}
