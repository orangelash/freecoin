\chapter{Implementação}
\label{chap:implmentacao}

\section{Ferramentas e Tecnologias utilizadas}
\label{sec:ferramentas}
O projeto foi desenvolvido de forma a ser usado num computador, para tal foi utilizada a linguagem \textit{Java}, compilada e testada no IDE \textit{NetBeans}. De forma a se encontrar disponível para todos os discentes podendo trabalhar em simultâneo e efetuando controlo de versões.


Desta forma, resulta uma aplicação que corre em linha de comandos (CLI) e um relatório concebido em \emph{Share} \LaTeX



\section{Escolhas de Implementação}
\label{sec:escolhas}

Numa fase inicial, foi necessário tomar algumas decisões fulcrais. Algumas destas decisões basearam-se na escolha da linguagem a ser usada, desta forma foi escolhida a linguagem \textit{Java} derivado a ser uma linguagem orienteada a objetos e também com a criação de uma interface CLI.

\section{Detalhes de Implementação}
\label{sec:detalhes}

\section{Manual de Instalação}
\label{sec:instalacao}

\section{Manual de Utilização}
\label{sec:utilizacao}

\section{Conclusão}
\label{sec:conc}